\documentclass[12pt,a4paper]{jsarticle}
\usepackage{bm}
\usepackage{amsmath,amssymb}
\usepackage{ascmac}
\def\vector#1{\mbox{\boldmath $#1$}}
%
\usepackage{graphicx,color}
\graphicspath{{../../SEM_practice/Patera/Accuracy/},{/home/ryo/Pictures/}}
%
%参考文献:biblatexを使う
\usepackage[backend=bibtex, style=numeric]{biblatex}
\addbibresource{TurbulenceModel.bib}
%
% \usepackage{xcolor}
% \usepackage[draft]{graphicx}
% \usepackage{booktabs} % 表の横罫線
% \usepackage{lscape}  % 表などを90度回転させる
%% 囲み枠
% \usepackage{tcolorbox}
\usepackage{longtable} % ページをまたぐ表の作成
\makeatletter %式番号を章ごとに
  \renewcommand{\theequation}{%
  \thesection.\arabic{equation}}
  \@addtoreset{equation}{section}
  \renewcommand{\thefigure}{% 図番号の付け方
	\thesection.\arabic{figure}}
	\@addtoreset{figure}{section}
  \renewcommand{\thetable}{% 表番号の付け方
	\thesection.\arabic{table}}
	\@addtoreset{table}{section}
\makeatother
\usepackage{subfig}
%
\title{乱流モデル勉強会(LESとRANS)}
\author{荒木亮}
\date{\today}

\begin{document}
%タイトル出力
\maketitle
%概要出力
\begin{abstract}
  「乱流モデル」のうちLESおよびRANSを対象に概説する.

  文章中のノーテーションは\cite{Kajishima}に基づいており,この注釈書として読むことを想定している.
\end{abstract}


\section{支配方程式}
\label{sec:BasicKnowledge}

流体の非圧縮条件と運動方程式であるNavier-Stokes方程式を示す.
\begin{equation}
  \frac{\partial u_i}{\partial x_i} = 0
  \label{eq:Incompressibility}
\end{equation}
\begin{equation}
  \frac{\partial u_i}{\partial t} +\frac{\partial (u_i u_j)}{\partial x_j} = -\frac{1}{\rho} \frac{\partial p}{\partial x_i} +\frac{\partial}{\partial x_j} (2 \nu D_{ij})
\label{eq:N-SEquation}
\end{equation}
ただし,粘性項をひずみ速度テンソル$D_{ij}$で書いている.
\begin{align}
  D_{ij} &= \frac{1}{2} \Big( \frac{\partial u_i}{\partial x_j} +\frac{\partial u_j}{\partial x_i} \Big)
  \label{eq:VelocityStrainTensor} \\
  \frac{\partial}{\partial x_j} (2 \nu D_{ij}) &= \nu \Big( \frac{\partial}{\partial x_j} \frac{\partial u_i}{\partial x_j} +\frac{\partial}{\partial x_j} \frac{\partial u_j}{\partial x_i} \Big) \nonumber \\
    &= \nu \frac{\partial^2 u_i}{\partial x_j \partial x_j}
\label{eq:DiffusionTerm}
\end{align}


\clearpage
\section{LES}
\label{sec:LES}


\subsection{フィルター操作}
\label{subsec:Filter}

\noindent{・フィルタ関数$G(y)$の性質}

%%%%%%%%%%%%%%%%%%%%%%%%%%%%%%%%%%%%%%%%%%%%%%
\begin{itembox}[l]{\cite{Kajishima}での定義}
  $G(y)$は$y=0$の近くで正の値をもち,$\displaystyle{\lim_{y \to \pm \infty} G(y) = 0}$となり,
  \begin{equation}
    \int_{-\infty}^\infty G(y) \mathrm{d}y =1 \tag{7.7}
    \label{eq:RegularofFilterFunctionKajishima}
  \end{equation}
  であるようなもの.
\end{itembox}
%%%%%%%%%%%%%%%%%%%%%%%%%%%%%%%%%%%%%%%%%%%%%%
%%%%%%%%%%%%%%%%%%%%%%%%%%%%%%%%%%%%%%%%%%%%%%
\begin{itembox}[l]{\cite{KidaYanase}での定義}
  $G(\vector{x})$は規格化条件
  \begin{equation}
    \int G(\vector{x}) \mathrm{d}\vector{x} =1 \tag{14.2}
    \label{eq:RegularofFilterFunctionKidaYanase}
  \end{equation}
  と対称条件
  \begin{equation}
    G(\vector{x}) =G(-\vector{x}) \tag{14.3}
    \label{eq:SymmetryofFilterFunctionKidaYanase}
  \end{equation}
  を満たすものとする.
\end{itembox}\\
%%%%%%%%%%%%%%%%%%%%%%%%%%%%%%%%%%%%%%%%%%%%%%


% \noindent{・ボックスフィルターのFourier変換\cite{Chiba}}
%
% 次式で定義するボックスフィルターのFourier変換を考える.
% \begin{align}
%   G(x) =\left\{ \begin{array}{ll}
%   1/\Delta  & (|x| \le \Delta/2) \\
%   0         & (|x| > \Delta/2) \\
% \end{array} \right.
% \label{eq:BoxFilter}
% \end{align}
%
% ヘビサイド関数$H(x)$を用意する.
% \begin{align}
%   H(x) =\left\{ \begin{array}{ll}
%   1  & (x > 0) \\
%   0  & (x < 0) \\
%   \end{array} \right.
%   \label{eq:HeavisideFunction}
% \end{align}
%
% これを用いてボックスフィルターを次のように書く.
% \begin{equation}
%   G(x) = \frac{1}{\Delta} \Big( H(x+\Delta/2) - H(x-\Delta/2) \Big)
% \label{eq:BoxFilterwithHeavisideFunction}
% \end{equation}
%
% $f(x)=H(x+1)-H(x-1)$のFourier変換は
% \begin{equation}
%   F(x) = \int_{-1}^1 \exp(-\mathrm{i}kx) \mathrm{d}x = \frac{2}{k} \sin(k)
%   % \label{eq:BoxFilterwithHeavisideFunction}
% \end{equation}
%
% より,$G(x) =\frac{1}{\Delta} f(\Delta x/2)$のFourier変換は次のようになる.
% \begin{equation}
%   \hat{G}(x) = \int_{-\Delta/2}^{\Delta/2} \frac{1}{\Delta} \exp(-\mathrm{i}kx\Delta/2) \mathrm{d}x = \frac{2}{\Delta k} \sin(\Delta k/2)
%   % \label{eq:BoxFilterwithHeavisideFunction}
% \end{equation}\\


\noindent{・フィルタ操作と微分操作の交換について}

物理量$f$にフィルタをかけ,GS(グリッドスケール)量$\bar{f}$とそこからの変動分$f'$に分ける.
\begin{equation}
  f = \bar{f} +f' \tag{7.7}
  \label{eq:Filter}
\end{equation}
これは物理量をGS以上,以下の2つのスケールに分離している.
しかし,フィルターが物理空間とは数空間のどちらか,あるいは両方でシャープでないため,厳密には分離できていない.

そのため,微分とフィルタの交換
\begin{equation}
  \overline{ \frac{\partial f}{\partial x} } = \frac{\partial \bar{f}}{\partial x}
  \label{eq:ExchangeofFilterDerivative}
\end{equation}
は厳密には成立しない.\\


\noindent{・LESの基礎方程式の導出}

Eq. \ref{eq:Incompressibility}とEq. \ref{eq:N-SEquation}にフィルタをかける.
フィルタ操作と微分操作が可換であるとして,
\begin{equation}
  \overline{\frac{\partial u_i}{\partial x_i}} = 0 \leftrightarrow \frac{\partial \bar{u}_i}{\partial x_i} = 0 \tag{7.13}
  \label{eq:FilteredIncompressibility}
\end{equation}
\begin{align}
  \overline{\frac{\partial u_i}{\partial t}} +\overline{\frac{\partial (u_i u_j)}{\partial x_j}} &= -\frac{1}{\rho} \overline{\frac{\partial p}{\partial x_i}} +\overline{\frac{\partial}{\partial x_j} (2 \nu D_{ij})} \leftrightarrow \nonumber \\
  \frac{\partial \bar{u}_i}{\partial t} +\frac{\partial (\bar{u}_i \bar{u}_j)}{\partial x_j} &= -\frac{1}{\rho} \frac{\partial \bar{p}}{\partial x_i} +\frac{\partial}{\partial x_j} (-\tau_{ij} +2 \nu \overline{D}_{ij}) \tag{7.16}
\label{eq:FilteredN-SEquation}
\end{align}

Eq. \ref{eq:FilteredN-SEquation}において,両辺第一項は自明.

左辺第二項を計算する.
\begin{align}
  \overline{\frac{\partial (u_i u_j)}{\partial x_j}} &= \frac{\partial \overline{(u_i u_j)}}{\partial x_j} \nonumber \\
    &= \frac{\partial (\bar{u}_i \bar{u}_j)}{\partial x_j} +\underbrace{ \frac{\partial \overline{(u_i u_j)}}{\partial x_j} -\frac{\partial (\bar{u}_i \bar{u}_j)}{\partial x_j} }_{これを右辺にまとめる}
\label{eq:FilteredAdvectionTerm}
\end{align}

\begin{align}
  -\Big[ \frac{\partial \overline{(u_i u_j)}}{\partial x_j} -\frac{\partial (\bar{u}_i \bar{u}_j)}{\partial x_j} \Big] +\overline{\frac{\partial}{\partial x_j} (2 \nu D_{ij})} &= \frac{\partial}{\partial x_j} \Big[ -(\overline{u_i u_j} -\bar{u}_i \bar{u}_j) +2 \nu \overline{D}_{ij} \Big] \nonumber \\
    &= \frac{\partial}{\partial x_j} (-\tau_{ij} +2 \nu \overline{D}_{ij})
\label{eq:FilteredDiffusionTerm}
\end{align}

ここで導入した$\tau_{ij}=\overline{u_i u_j} -\bar{u}_i \bar{u}_j$をSGS応力とよぶ.\\

\noindent{・$\tau_{ij}$の分解}

SGS応力$\tau_{ij}$のうち,2次の相関を含む第一項を$u_i = \bar{u}_i +u_i'$で分解する.
\begin{align}
  \tau_{ij} &= \overline{u_i u_j} -\bar{u}_i \bar{u}_j \nonumber \\
    &= \overline{(\bar{u}_i +u_i') (\bar{u}_j +u_j')} -\bar{u}_i \bar{u}_j \nonumber \\
    &= \Big[ \overline{\bar{u}_i \bar{u}_j} +\overline{\bar{u}_i u_j'} +\overline{u_i' \bar{u}_j} +\overline{u_i' u_j'} \Big] -\bar{u}_i \bar{u}_j \nonumber \\
    &= \Big[ \overline{\bar{u}_i \bar{u}_j} -\bar{u}_i \bar{u}_j \Big] +\Big[ \overline{\bar{u}_i u_j'} +\overline{u_i' \bar{u}_j} \Big] +\Big[ \overline{u_i' u_j'}  \Big] \nonumber \\
    &= L_{ij} +C{ij} +R_{ij} \tag{7.20}
\label{eq:DecompositionofSGSStress}
\end{align}

ただしLeonard項$L_{ij}$,クロス項$C_{ij}$,SGS Reynolds応力$R_{ij}$.\\


\noindent{・$\tau_{ij}$のGalilean不変性について}

$L_{ij}$,$C_{ij}$,$R_{ij}$それぞれを一定速度$U_i$で移動する系に移して検討する.

$u_i \to u_i^* = u_i +U_i$とし,$u_i'^*=u_i'$と$\bar{U}=U$に注意して計算する.
\begin{align}
  L_{ij}^* &= \overline{\bar{u}_i^* \bar{u}_j^*} -\bar{u}_i^* \bar{u}_j^* \nonumber \\
    &= \overline{(\bar{u}_i +U_i) (\bar{u}_j +U_j)} -(\bar{u}_i +U_i) (\bar{u}_j +U_j) \nonumber \\
    &= \overline{\bar{u}_i \bar{u}_j} +\bar{\bar{u}}_i U_j +U_i \bar{\bar{u}}_j +U_i U_j -\bar{u}_i \bar{u}_j -\bar{u}_i U_j -U_i \bar{u}_j -U_i U_j \nonumber \\
    &= \overline{\bar{u}_i \bar{u}_j} -\bar{u}_i \bar{u}_j +\big( \bar{\bar{u}}_i -\bar{u}_i \big) U_j +U_i \big( \bar{\bar{u}}_j -\bar{u}_j \big) \nonumber \\
    &\ne L_{ij}
\label{eq:GalileiTransformofL} \\
  C_{ij}^* &= \overline{\bar{u}_i^* u_j'^*} +\overline{u_i'^* \bar{u}_j^*} \nonumber \\
    &= \overline{(\bar{u}_i +U_i) u_j'} +\overline{u_i' (\bar{u}_j +U_j)} \nonumber \\
    &= \overline{\bar{u}_i u_j'} +U_i \overline{u_j'} +\overline{u_i' \bar{u}_j} +\overline{u_i'} U_j \nonumber \\
    &\ne C_{ij}
\label{eq:GalileiTransformofC} \\
  R_{ij}^* &= \overline{u_i'^* u_j'^*} \nonumber \\
    &= \overline{u_i' u_j'} \nonumber \\
    &= R_{ij}
\label{eq:GalileiTransformofR}
\end{align}

より,Reynolds応力項$R_{ij}$のみがGalilean不変性を満たす.

\subsection{Smagorinskyモデル}
\label{subsec:SmagorinskyModel}

Eq. \ref{eq:FilteredN-SEquation}に示すフィルターされたN-S式に$\bar{u}_i$を乗じて総和規約を取ることで,流れ場のエネルギ輸送方程式を導出する.
\begin{equation}
  \frac{\partial \bar{u}_i}{\partial t} +\frac{\partial (\bar{u}_i \bar{u}_j)}{\partial x_j} = -\frac{1}{\rho} \frac{\partial \bar{p}}{\partial x_i} +\frac{\partial}{\partial x_j} (-\tau_{ij} +2 \nu \overline{D}_{ij}) \tag{7.16}
\label{eq:FilteredN-SEquation}
\end{equation}
\begin{align}
  左辺 \cdot \bar{u}_i &= \frac{\partial \bar{u}_i}{\partial t} \bar{u}_i +\frac{\partial (\bar{u}_i \bar{u}_j)}{\partial x_j} \bar{u}_i \nonumber \\
    &= \Big( \frac{\partial}{\partial t} +\bar{u}_j \frac{\partial}{\partial x_j} \Big) \Big( \frac{1}{2}\bar{u}_i \bar{u}_i \Big) \nonumber \\
    &= \frac{ \overline{\mathrm{D}} k_{\mathrm{GS}} }{ \overline{\mathrm{D}} t}
\label{eq:L_sideofkgs}
\end{align}
\begin{align}
  右辺 \cdot \bar{u}_i &= -\frac{1}{\rho} \frac{\partial \bar{p}}{\partial x_i} \bar{u}_i +\frac{\partial}{\partial x_j} (-\tau_{ij} +2 \nu \overline{D}_{ij}) \bar{u}_i \nonumber \\
    第一項 &: -\frac{1}{\rho} \frac{\partial \bar{p}}{\partial x_i} \bar{u}_i = \frac{\partial}{\partial x_j} \Big( -\frac{1}{\rho} \bar{p}\bar{u}_j \Big) \nonumber \\
    第二項 &: \frac{\partial (-\tau_{ij})}{\partial x_j} \bar{u}_i = \frac{\partial}{\partial x_j} (-\tau_{ij} \bar{u}_i) -(-\tau_{ij}) \frac{\partial \bar{u}_i}{\partial x_j} \nonumber \\
    第三項 &: \frac{\partial (2 \nu \overline{D}_{ij}) }{\partial x_j} \bar{u}_i = \frac{\partial}{\partial x_j} (2 \nu \overline{D}_{ij} \bar{u}_i) -2 \nu \overline{D}_{ij} \frac{\partial \bar{u}_i}{\partial x_j} \nonumber \\
      &~~~~~~~~~~~~~~~~~~= \frac{\partial}{\partial x_j} (\nu \frac{\partial \bar{u}_i}{\partial x_j} \bar{u}_i) -\nu \frac{\partial \bar{u}_i}{\partial x_j} \frac{\partial \bar{u}_i}{\partial x_j} \nonumber \\
  \therefore 右辺 \cdot \bar{u}_i &= \frac{\partial}{\partial x_j} \Big( -\frac{1}{\rho}\bar{p}\bar{u}_j -\tau_{ij}\bar{u}_i +\nu \frac{\partial k_{\mathrm{GS}}}{\partial x_j} \Big) +\tau_{ij} \overline{D}_{ij} -\bar{\varepsilon}_{\mathrm{GS}}
  % &= \frac{\partial}{\partial x_j} \Big( -\frac{1}{\rho}\bar{p}\bar{u}_j -\tau_{ij}\bar{u}_i +\nu \frac{\partial \bar{u}_i}{\partial x_j} \bar{u}_i \Big) +\tau_{ij} \frac{\partial \bar{u}_{i}}{\partial x_j} -\nu \frac{\partial \bar{u}_i}{\partial x_j} \frac{\partial \bar{u}_i}{\partial x_j} \nonumber \\
\label{eq:R_sideofkgs}
\end{align}

以上をまとめて
\begin{equation}
  \Big( \frac{\partial}{\partial t} +\bar{u}_j \frac{\partial}{\partial x_j} \Big) k_{\mathrm{GS}} = \frac{\partial}{\partial x_j} \Big( -\frac{1}{\rho}\bar{p}\bar{u}_j -\tau_{ij}\bar{u}_i +\nu \frac{\partial k_{\mathrm{GS}}}{\partial x_j} \Big) +\tau_{ij} \overline{D}_{ij} -\bar{\varepsilon}_{\mathrm{GS}} \tag{7.29}
\label{eq:GSScaleEnergyEquation}
\end{equation}

ここで$\tau_{ij}\overline{D}_{ij}$がSGSスケールへのエネルギ輸送率となる.\\

SGSスケールに送られるエネルギ$\tau_{ij} \overline{D}_{ij}$と,SGSスケールで散逸するエネルギ$\varepsilon_{\mathrm{SGS}}$が釣り合うことを要請する.
これは局所平衡の仮定(エネルギが瞬時に散逸する仮定)である.
\begin{equation}
  \tau_{ij} \overline{D}_{ij} +\varepsilon_{\mathrm{SGS}} = 0 \tag{7.32}
\label{eq:LocalEquilibrium}
\end{equation}

SGSスケールでのエネルギ散逸$\varepsilon_{\mathrm{SGS}}$は,散逸項$\varepsilon$から導出できる.
\begin{align}
  \varepsilon &= \nu \frac{\partial u_i}{\partial x_j} \frac{\partial u_i}{\partial x_j} \nonumber \\
  \varepsilon_{\mathrm{SGS}} &= \bar{\varepsilon} -\varepsilon_{\mathrm{GS}} = \nu \overline{ \frac{\partial u_i}{\partial x_j} \frac{\partial u_i}{\partial x_j} } -\nu \frac{\partial \bar{u}_i}{\partial x_j} \frac{\partial \bar{u}_i}{\partial x_j} \tag{7.31}
\label{eq:SGSDifusion}
\end{align}

結局,モデル化に必要な量のうちSGS応力$\tau_{ij}$のみが2次の相関を含む.
これを1次の量で表したい.
渦粘性近似を導入する.
\begin{align}
  \tau_{ij} &= \overline{u_i u_j} -\bar{u}_i \bar{u}_j \nonumber \\
  \tau_{ij} &\fallingdotseq -2\nu_e \overline{D}_{ij} +\frac{1}{3} \delta_{ij} \tau_{kk}
\label{eq:EddyViscosityApproximation}
\end{align}
ここで,等方成分$\displaystyle{ \frac{1}{3} \delta_{ij} \tau_{kk} }$は圧力項にくりこまれる.
非等方成分のみ表したのがEq. 7.33である.

この仮定をEq. \ref{eq:FilteredN-SEquation}に導入する.
\begin{align}
  \frac{\partial \bar{u}_i}{\partial t} +\frac{\partial (\bar{u}_i \bar{u}_j)}{\partial x_j} &= -\frac{1}{\rho} \frac{\partial \bar{p}}{\partial x_i} +\frac{\partial}{\partial x_j} (-\tau_{ij} +2 \nu \overline{D}_{ij}) \nonumber \\
    &= -\frac{1}{\rho} \frac{\partial \bar{p}}{\partial x_i} +\frac{\partial}{\partial x_j} \bigl\{ (-2\nu_e \overline{D}_{ij} +\frac{1}{3} \delta_{ij} \tau_{kk}) +2 \nu \overline{D}_{ij} \bigr\} \nonumber \\
    &= -\frac{1}{\rho} \frac{\partial (\bar{p}+\rho \tau_{kk}/3 )}{\partial x_i} +\frac{\partial}{\partial x_j} \{2 (\nu +\nu_e) \overline{D}_{ij}\} \nonumber \\
    &= -\frac{1}{\rho} \frac{\partial \bar{P}}{\partial x_i} +\frac{\partial}{\partial x_j} \{2 (\nu +\nu_e) \overline{D}_{ij}\} \tag{7.34}
\label{eq:SmagorinskyNSEquation}
\end{align}


\subsection{Smagorinsky定数}
\label{subsec:SmagorinskyConstant}

前節で導入した渦動粘性係数$\nu_e$を具体的な形で求めたい.
次元解析より$[\nu_e] =[\mathrm{L}^2/\mathrm{T}] =[\mathrm{L}/\mathrm{T}][\mathrm{L}]$なので,速度スケール$q$と長さスケール$l$で表せる.
\begin{equation}
  \nu_e = C_v ql \tag{7.35}
\label{eq:VortexKinematicViscosity}
\end{equation}

また,SGSスケールでのエネルギ散逸$\varepsilon_{\mathrm{SGS}}$は,$[\varepsilon_{\mathrm{SGS}}] =[\mathrm{L}^2/\mathrm{T}^3] =[\mathrm{L}/\mathrm{T}]^3 [1/\mathrm{L}]$の次元を持つ.
\begin{equation}
  \varepsilon_{\mathrm{SGS}} = C_\varepsilon \frac{q^3}{l}
  \label{eq:DiffusionSGS}
\end{equation}

ここから$q$を消去する.
\begin{equation}
  \frac{\varepsilon_{\mathrm{SGS}}}{\nu_e^3} = \frac{C_\varepsilon}{C_v^3} \frac{1}{l^4}
\end{equation}


また,2式Eq. \ref{eq:VortexKinematicViscosity}とEq. \ref{eq:DiffusionSGS}を,Eq. \ref{eq:LocalEquilibrium}で結びつける.
\begin{align}
  \varepsilon_{\mathrm{SGS}} &= -\tau_{ij} \overline{D}_{ij} \nonumber \\
    &= 2 \nu_e \overline{D}_{ij} \overline{D}_{ij} \nonumber \\
  \therefore \frac{\varepsilon_{\mathrm{SGS}}}{\nu_e^3} &= \frac{2 \overline{D}_{ij} \overline{D}_{ij}}{\nu_e^2}
\end{align}

よって,
\begin{equation}
  \frac{C_\varepsilon}{C_v^3} \frac{1}{l^4} = \frac{2 \overline{D}_{ij} \overline{D}_{ij}}{\nu_e^2}
\end{equation}

$\nu_e$について解き,
\begin{align}
  \nu_e^2 &= \frac{C_v^3}{C_\varepsilon} l^4 2 \overline{D}_{ij} \overline{D}_{ij} \nonumber \\
  \nu_e &= \sqrt{\frac{C_v^3}{C_\varepsilon}} \cdot l^2 \cdot \sqrt{2 \overline{D}_{ij} \overline{D}_{ij}} \nonumber \\
    &= C_s^2 \cdot \Delta^2 \cdot |\overline{D}| \tag{7.37}
\end{align}

ここで$C_s$がSmagorinsky定数であり,Kolmogorovのエネルギスペクトル指数から理論値を計算することができる.\\

これによって得られた普遍的なSmagorinsky定数$C_{\varepsilon}$から計算した$\nu_e$は常に正で,GSスケールからSGSスケールへエネルギが流れることを意味する.
これは,逆カスケードや壁近傍の流れに対応できない.

これを改良するため,壁法則を導入する,$\tau_{ij}$の分解にさらにモデルを適用する,Smagorinsky定数$C_{\varepsilon}$を時空間変数によるものとする,などの理論がある.

\clearpage
\section{RANS}
\label{sec:RANS}


\subsection{Reynolds平均}
\label{subsec:ReynoldsAverage}

\noindent{・平均操作について}

流れ場が統計的に定常なら時間平均,空間的に一様なら空間平均を取ればよい.
より一般にはアンサンブル平均を用いる.

流速$u_i$を平均$\bar{u}_i$と乱れ$u_i'$に分解する.
\begin{equation}
  u_i = \bar{u}_i +u_i'
\label{eq:ReynoldsDecomposition}
\end{equation}

ここで,平均値も緩やかに変動することから,「平均値と変動値の積(=相関)の平均」は0にならず,「平均値のアンサンブル平均」は平均値と一致しない.
\begin{equation}
  \overline{u_i' \bar{u}_j} \ne 0, ~~~~ \bar{\bar{u}}_i \ne \bar{u}_i \tag{6.1}
  \label{eq:AverageofAverage}
\end{equation}

しかし,\underline{上式が成り立つような平均操作}を定義し,これをReynolds平均とよぶ.
\begin{equation}
  \overline{u_i'} = 0, ~~~~ \overline{u_i' \bar{u}_j} = 0, ~~~~ \bar{\bar{u}}_i = \bar{u}_i \tag{6.2}
  \label{eq:CharacteristicsofReynoldsAverage}
\end{equation}

このとき平均操作と微分操作の交換が許される.
\begin{equation}
  \overline{ \frac{\partial u}{\partial x} } = \frac{\partial \bar{u}}{\partial x}
  \label{eq:ExchangeofAverageDerivative}
\end{equation}

\noindent{・RANSの基礎方程式の導出}

Eq. \ref{eq:Incompressibility}とEq. \ref{eq:N-SEquation}をReynolds平均する.
\begin{equation}
  \overline{\frac{\partial u_i}{\partial x_i}} = 0 \leftrightarrow \frac{\partial \bar{u}_i}{\partial x_i} = 0 \tag{6.4}
  \label{eq:AveragedIncompressibility}
\end{equation}
\begin{align}
  \overline{\frac{\partial u_i}{\partial t}} +\overline{\frac{\partial (u_i u_k)}{\partial x_k}} &= -\frac{1}{\rho} \overline{\frac{\partial p}{\partial x_i}} +\overline{\frac{\partial}{\partial x_k} (2 \nu D_{ik})} \leftrightarrow \nonumber \\
  \frac{\partial \bar{u}_i}{\partial t} +\frac{\partial (\bar{u}_i \bar{u}_k)}{\partial x_k} &= -\frac{1}{\rho} \frac{\partial \bar{p}}{\partial x_i} +\frac{\partial}{\partial x_k} (-\tau_{ik} +2 \nu \overline{D}_{ik}) \tag{6.6}
\label{eq:AveragedN-SEquation}
\end{align}
%
Eq. \ref{eq:AveragedN-SEquation}はEq. \ref{eq:FilteredN-SEquation}と同じ計算で得られるが,左辺第二項のみ異なる.
\begin{equation}
  \overline{u_i u_j} = \bar{u}_i \bar{u}_j +\overline{u_i' u_j'} \tag{6.8}
  \label{eq:NonlinearTermofReynoldsAverage}
\end{equation}
%
よってReynolds応力$\tau_{ij}$は次式で定まる.
\begin{equation}
  \tau_{ij} = \overline{u_i' u_j'} \tag{6.9}
  \label{eq:ReynoldsStress}
\end{equation}


\subsection{Reynolds応力方程式}
\label{subsec:ReynoldsStressEquation}

元の連続の式Eq. \ref{eq:Incompressibility}からReynolds平均された方程式Eq. \ref{eq:AveragedIncompressibility}を除すことで,「変動速度の非圧縮条件」を得る.
\begin{equation}
  \frac{\partial u_i'}{\partial x_i} = 0 \tag{6.10}
  \label{eq:N-S-AveragedIncompressibility}
\end{equation}

元のNavier-Stokes方程式Eq. \ref{eq:N-SEquation}からReynolds平均された方程式Eq. \ref{eq:AveragedN-SEquation}を除すことで,「変動速度のNavier-Stokes方程式」を得る.
\begin{align*}
  時間微分項 &: \frac{\partial u_i}{\partial t} -\frac{\partial \bar{u}_i}{\partial t} = \frac{\partial u_i'}{\partial t} \\
  移流項    &: \frac{\partial u_i u_k}{\partial x_k} -\frac{\partial (\bar{u}_i \bar{u}_k)}{\partial x_k} = \frac{\partial}{\partial x_k}  (u_i' u_k' +\bar{u}_i u_k' +u_i' \bar{u}_k) \\
  圧力項    &: -\frac{1}{\rho} \frac{\partial p}{\partial x_i} +\frac{1}{\rho} \frac{\partial \bar{p}}{\partial x_i} = -\frac{1}{\rho} \frac{\partial p'}{\partial x_i} \\
  粘性項    &: \frac{\partial}{\partial x_k} (2 \nu D_{ik}) -\frac{\partial}{\partial x_k} (-\tau_{ik} +2 \nu \overline{D}_{ik}) = \frac{\partial}{\partial x_k} (\tau_{ik} +2 \nu D_{ik}')
\end{align*}
\begin{equation}
  \frac{\partial u_i'}{\partial t} +\bar{u}_k \frac{\partial u_i'}{\partial x_k} = -\frac{1}{\rho} \frac{\partial p'}{\partial x_i} +\frac{\partial}{\partial x_k} (-u_i' u_k' -\bar{u}_i u_k' +\tau_{ik} +2 \nu D_{ik}') \tag{6.11}
  \label{eq:N-S-AveragedN-SEquation}
\end{equation}

Eq. \ref{eq:N-S-AveragedN-SEquation}に$u_j'$を乗じ,Reynolds平均をとり,添字$i$と$j$を入れ替えて足すことでReynolds応力方程式を導出する.
Eq. \ref{eq:N-S-AveragedN-SEquation}の各項に番号をふって再掲する.
\begin{equation*}
  \frac{\partial u_i'}{\partial t}_{\textcircled{\scriptsize 1}} +\bar{u}_k \frac{\partial u_i'}{\partial x_k}_{\textcircled{\scriptsize 2}} = -\frac{1}{\rho} \frac{\partial p'}{\partial x_i}_{\textcircled{\scriptsize 3}} +\frac{\partial}{\partial x_k} (-u_i' u_{k'\textcircled{\scriptsize 4}} -\bar{u}_i u_{k'\textcircled{\scriptsize 5}} +\tau_{ik \textcircled{\scriptsize 6}} +2 \nu D'_{ik \textcircled{\scriptsize 7}})
\end{equation*}
丸数字をつけた各項を計算する.
\begin{align*}
  \textcircled{\scriptsize 1} &: \overline{ \frac{\partial u_i'}{\partial t} u_j' } +\overline{ \frac{\partial u_j'}{\partial t} u_i' } = \frac{\partial}{\partial t} \overline{u_i' u_j'} \\
  \textcircled{\scriptsize 2} &: \overline{ \bar{u}_k \frac{\partial u_i'}{\partial x_k} u_j' } +\overline{ \bar{u}_k \frac{\partial u_j'}{\partial x_k} u_i' } = \bar{u}_k \frac{\partial}{\partial x_k} \overline{u_i' u_j'} \\
  \textcircled{\scriptsize 3} &: -\overline{ \frac{1}{\rho} \frac{\partial p'}{\partial x_i} u_j' } -\overline{ \frac{1}{\rho} \frac{\partial p'}{\partial x_j} u_i' } = - \frac{1}{\rho} \Big( \overline{ \frac{\partial p'}{\partial x_i} u_j' } +\overline{ \frac{\partial p'}{\partial x_j} u_i' } \Big) \\
  \textcircled{\scriptsize 4} &: \overline{ \frac{\partial (-u_i' u_{k'})}{\partial x_k} u_j' } +\overline{ \frac{\partial (-u_j' u_{k'})}{\partial x_k} u_i' } = -\frac{\partial}{\partial x_k} \overline{u_i' u_j' u_k'} \\
  \textcircled{\scriptsize 5} &: \overline{ \frac{\partial (-\bar{u}_i' u_{k'})}{\partial x_k} u_j' } +\overline{ \frac{\partial (-\bar{u}_j' u_{k'})}{\partial x_k} u_i' } = -\overline{u_j' u_k'} \frac{\partial \bar{u}_i}{\partial x_k} -\overline{u_i' u_k'} \frac{\partial \bar{u}_j}{\partial x_k} \\
  \textcircled{\scriptsize 6} &: \overline{ \frac{\partial (\overline{u_i' u_k'})}{\partial x_k} u_j' } +\overline{ \frac{\partial (\overline{u_j' u_k'})}{\partial x_k} u_i' } = 0 ~(?) \\
  \textcircled{\scriptsize 7} &: \overline{2 \nu \frac{\partial}{\partial x_k} \Big[ \frac{1}{2} \Big( \frac{\partial u_i'}{\partial x_k} +\frac{\partial u_k'}{\partial x_i} \Big) \Big] u_j' } +\overline{2 \nu \frac{\partial}{\partial x_k} \Big[ \frac{1}{2} \Big( \frac{\partial u_j'}{\partial x_k} +\frac{\partial u_k'}{\partial x_j} \Big) \Big] u_i' } \\
    &~~~ = \nu \frac{\partial^2}{\partial x_k^2} \overline{u_i' u_j'} -2\nu \overline{ \frac{\partial u_i'}{\partial x_k} \frac{\partial u_i'}{\partial x_k} }
\end{align*}
各項を一旦まとめ,置き換えられる部分を$\tau_{ij}$で書く.
\begin{align*}
  \frac{\partial}{\partial t} \overline{u_i' u_j'} +\bar{u}_k \frac{\partial}{\partial x_k} \overline{u_i' u_j'} &= - \frac{1}{\rho} \Big( \overline{ \frac{\partial p'}{\partial x_i} u_j' } +\overline{ \frac{\partial p'}{\partial x_j} u_i' } \Big) -\frac{\partial}{\partial x_k} \overline{u_i' u_j' u_k'} -\overline{u_j' u_k'} \frac{\partial \bar{u}_i}{\partial x_k} -\overline{u_i' u_k'} \frac{\partial \bar{u}_j}{\partial x_k} \\
    & ~~~~+\nu \frac{\partial^2}{\partial x_k^2} \overline{u_i' u_j'} -2\nu \overline{ \frac{\partial u_i'}{\partial x_k} \frac{\partial u_i'}{\partial x_k} } \\
    %
  \Big( \frac{\partial}{\partial t} +\bar{u}_k \frac{\partial}{\partial x_k} \Big) \tau_{ij} &= - \frac{1}{\rho} \Big( \overline{ \frac{\partial p'}{\partial x_i} u_j' } +\overline{ \frac{\partial p'}{\partial x_j} u_i' } \Big) -\frac{\partial}{\partial x_k} \overline{u_i' u_j' u_k'} -\tau_{jk} \frac{\partial \bar{u}_i}{\partial x_k} -\tau_{ik} \frac{\partial \bar{u}_j}{\partial x_k} \\
    & ~~~~+\nu \Big( \frac{\partial^2 \tau_{ij}}{\partial x_k^2} -2 \overline{ \frac{\partial u_i'}{\partial x_k} \frac{\partial u_i'}{\partial x_k} } \Big)
\end{align*}
各項を表現しなおす.
\begin{align*}
  左辺 &: \Big( \frac{\partial}{\partial t} +\bar{u}_k \frac{\partial}{\partial x_k} \Big) \tau_{ij} = \frac{\overline{\mathrm{D}} \tau_{ij}}{\overline{\mathrm{D}} t} \\
  圧力項 &: \frac{1}{\rho} \overline{ p' \Big( \frac{\partial u_i'}{\partial x_j} +\frac{\partial u_j'}{\partial x_i} \Big) } -\frac{1}{\rho} \Big( \frac{\partial}{\partial x_j} \overline{u_i' p'} +\frac{\partial}{\partial x_i} \overline{u_j' p'} \Big) \\
    &=  \frac{1}{\rho} \overline{ p' \Big( \frac{\partial u_i'}{\partial x_j} +\frac{\partial u_j'}{\partial x_i} \Big) } -\frac{\partial}{\partial x_k} \Big( \frac{1}{\rho} \overline{p'(\delta_{jk}u_i' +\delta_{ik}u_j')} \Big)
\end{align*}

以上より$\partial/\partial x_k$でまとめるとReynolds応力方程式を得る.
\begin{align}
  \frac{\overline{\mathrm{D}} \tau_{ij}}{\overline{\mathrm{D}} t} = -\tau_{ik} \frac{\partial \bar{u}_j}{\partial x_k} &-\tau_{jk} \frac{\partial \bar{u}_i}{\partial x_k} +\frac{1}{\rho} \overline{ p' \Big( \frac{\partial u_i'}{\partial x_j} +\frac{\partial u_j'}{\partial x_i} \Big) } -2 \nu \overline{ \frac{\partial u_i'}{\partial x_k} \frac{\partial u_i'}{\partial x_k} } \nonumber \\
    &+\frac{\partial}{\partial x_k} \Big( -\overline{u_i' u_j' u_k'} -\frac{1}{\rho} (\overline{p'u_i'} \delta_{jk} +\overline{p'u_j'} \delta_{ik}) +\frac{\partial \tau_{ij}}{\partial x_k} \Big)
  \label{eq:ReynoldsStressEquation_foot}
\end{align}

\begin{equation}
  \therefore
  \frac{\overline{\mathrm{D}} \tau_{ij}}{\overline{\mathrm{D}} t} = P_{ij} +\Pi_{ij} -\varepsilon_{ij} +\frac{\partial J_{ijk}}{\partial x_k}
  \label{eq:ReynoldsStressEquation_Tensor} \tag{6.12}
\end{equation}

各項は,それぞれReynolds応力の
\begin{itemize}
 \item[-] 生成項 $P_{ij}$
 \item[-] 圧力-ひずみ相関項(再分配項) $\Pi_{ij}$
 \item[-] 散逸項 $\varepsilon_{ij}$
 \item[-] (速度変動・圧力変動・粘性による)拡散項 $J_{ijk} =J_{(T)ijk} +J_{(P)ijk} +J_{(V)ijk}$
\end{itemize}
である.

2次の変動速度相関であるReynolds応力の支配方程式にさらに高次(3次)の相関項が現れるため,この方程式は閉じていない.
以降,3次変動速度,4次変動速度…の式を導いても,より高次の相関項が現れてしまう(乱流の完結の問題).\\

Reynolds応力方程式を「閉じた」形にするためのモデルとして,次の2つがある.
\begin{itemize}
 \item[-] 渦粘性モデル:~Reynolds応力自身をモデル化し,渦粘性係数を求める問題に帰着する
 \item[-] Reynolds応力方程式モデル:~Reynolds応力方程式の高次項(圧力-ひずみ相関項$\Pi_{ij}$,散逸項$\varepsilon_{ij}$と拡散項$J_{(T)ijk}$,$J_{(P)ijk}$)をモデル化する
\end{itemize}

\subsection{渦粘性モデル}
\label{subsec:EddyViscosityModel}

「分子粘性応力$=$粘性係数$\times$速度勾配」のアナロジーから,Reynolds応力テンソルを渦粘性係数とひずみ速度テンソルの積と考える.
\begin{equation}
  \rho \tau_{ij} = \frac{2}{3}\delta_{ij}\rho k -2\mu_{T} \overline{D}_{ij}
  \label{eq:EddyViscosityModel}
\end{equation}

右辺第一項は,$i=j$として総和をとったとき,左辺陶片が一致するように定めた.
\begin{itemize}
 \item[-] 左辺:$\rho \tau_{ii} =2\rho k$
 \item[-] 右辺:$\frac{2}{3}\times3\rho k -0 =2\rho k$
\end{itemize}
この仮定をEq. 6.4に導入する.
\begin{align}
  \frac{\partial \bar{u}_i}{\partial t} +\frac{\partial (\bar{u}_i \bar{u}_j)}{\partial x_j} &= -\frac{1}{\rho} \frac{\partial \bar{p}}{\partial x_i} +\frac{\partial}{\partial x_j} (-\tau_{ij} +2 \nu \overline{D}_{ij}) \nonumber \\
    &= -\frac{1}{\rho} \frac{\partial \bar{p}}{\partial x_i} +\frac{\partial}{\partial x_j} \bigl\{ -(\frac{2}{3}\delta_{ij}k -2\nu_T \overline{D}_{ij}) +2 \nu \overline{D}_{ij} \bigr\} \nonumber \\
    &= -\frac{1}{\rho} \frac{\partial (\bar{p}+2\rho k/3 )}{\partial x_i} +\frac{\partial}{\partial x_j} \{2 (\nu +\nu_T) \overline{D}_{ij}\} \nonumber \\
    &= -\frac{1}{\rho} \frac{\partial \bar{P}}{\partial x_i} +\frac{\partial}{\partial x_j} \{2 (\nu +\nu_T) \overline{D}_{ij}\} \tag{6.21}
\label{eq:EddyViscosityNSEquation}
\end{align}
圧力にReynolds応力の等方成分が加わり,動粘性係数に\underline{渦}動粘性係数が加わった.
結局,\underline{渦動粘性係数$\nu_T$をどうにかして既知の値から求めれば良いことになる.}
これを定めるために考える「乱流諸量の輸送方程式」の数により,$n$方程式モデルという.

\subsection{2方程式モデル}
\label{subsec:TwoEquationModel}

渦動粘性係数$\nu_T$を求めるため,\underline{2つの量の輸送方程式}を考える($\leftrightarrow$2つのパラメータから$\nu_T$を決定する).
パラメータとして次のようなものが考えられる.
\begin{itemize}
 \item[-] 変動流の運動エネルギの輸送方程式を考え,$k$(あるいはその平方根=速度)をパラメータとする.
 \item[-] エネルギ散逸率の輸送方程式を考え,$\varepsilon$をパラメータとする.
 \item[-] 時間の逆数である$\omega$をパラメータとする.
\end{itemize}
以下では最も代表的な$k-\varepsilon$モデルについて述べる.

\subsubsection{$k-\varepsilon$モデル}
\label{subsubsec:k-epsilonModel}

変動流の運動エネルギ$k$とエネルギ散逸率$\varepsilon$をパラメータとする.

Eq. \ref{eq:ReynoldsStressEquation_Tensor}で$i=j$として総和規約を用いることで,変動流エネルギ$k=\overline{\hat{u}_j \hat{u}_j}/2$の支配方程式が得られる.
\begin{equation}
  \frac{\overline{\mathrm{D}} k}{\overline{\mathrm{D}} t} = P_k -\varepsilon +\frac{\partial J_{(k)ij}}{\partial x_j}
  \label{eq:RANS_EnergyEquation} \tag{6.30}
\end{equation}
\begin{align}
  P_k &= \frac{1}{2}P_{ii} =-\tau_{ik} \frac{\partial \bar{u}_j}{\partial x_k} \tag{6.31} \\
  \varepsilon &= \frac{1}{2} \varepsilon_{ii} = \nu \overline{\Big( \frac{\partial \bar{u}_j}{\partial x_k} \Big)^2} \tag{6.32} \\
  J_{(k)j} &= \frac{1}{2}J_{iij} = -\overline{u_i' u_i' u_j'} -\frac{1}{\rho} \overline{p'u_k'} +\frac{\partial k}{\partial x_j} \tag{6.33~-~6.35}
\end{align}
$\Pi$が消えていることに注意.
この項はエネルギの再分配に関わっている.\\

$\varepsilon$の支配方程式を求める.
Eq. \ref{eq:N-S-AveragedN-SEquation}を$x_m$で偏微分する.
\begin{equation*}
  \frac{\partial u_i'}{\partial t} +\bar{u}_k \frac{\partial u_i'}{\partial x_k} = -\frac{1}{\rho} \frac{\partial p'}{\partial x_i} +\frac{\partial}{\partial x_k} (-u_i' u_k' -\bar{u}_i u_k' +\tau_{ik} +2 \nu D_{ik}') \tag{6.11}
\end{equation*}
\begin{align*}
  左辺 &= \Big( \frac{\partial}{\partial t} +\bar{u}_k \frac{\partial}{\partial x_k} \Big) \frac{\partial u_i'}{\partial x_m} + \frac{\partial \bar{u}_k}{\partial x_m} \frac{\partial u_i'}{\partial x_k} \\
  右辺 &= -\frac{1}{\rho} \frac{\partial^2 p'}{\partial x_i \partial x_m} +\frac{\partial}{\partial x_k \partial x_m} (-u_i' u_k' -\bar{u}_i u_k' +\tau_{ik} +2 \nu D_{ik}') \\
    &= -\frac{1}{\rho} \frac{\partial^2 p'}{\partial x_i \partial x_m} +\frac{\partial}{\partial x_k \partial x_m} (-u_i' u_k' +\tau_{ik}) -\frac{\partial \bar{u}_i}{\partial x_k} \frac{\partial u_k'}{\partial x_m} -\frac{\partial^2 \bar{u}_i}{\partial x_k \partial x_m} u_k' +\nu \frac{\partial^3 u_i'}{\partial x_k \partial x_k \partial x_m}
\end{align*}
\begin{align*}
  \Big( \frac{\partial}{\partial t} +\bar{u}_k \frac{\partial}{\partial x_k} \Big) \frac{\partial u_i'}{\partial x_m} = &-\frac{\partial^2 \bar{u}_i}{\partial x_k \partial x_m} u_k' -\frac{\partial \bar{u}_k}{\partial x_m} \frac{\partial u_i'}{\partial x_k} -\frac{\partial \bar{u}_i}{\partial x_k} \frac{\partial u_k'}{\partial x_m} \\
    &-\frac{\partial}{\partial x_k \partial x_m} (u_i' u_k' -\tau_{ik}) -\frac{1}{\rho} \frac{\partial^2 p'}{\partial x_i \partial x_m} +\nu \frac{\partial^3 u_i'}{\partial x_k \partial x_k \partial x_m}
\end{align*}
%
これに,
\begin{itemize}
  \item[-] 梶島\cite{Kajishima}:~$\displaystyle{2\nu \frac{\partial u_i'}{\partial x_m} \frac{\partial}{\partial t} \Big[ \frac{\partial u_i'}{\partial x_m} \Big] }$の演算を行って,Reynolds平均を施す.
  \item[-] 木田・柳瀬\cite{KidaYanase}:~$\displaystyle{\frac{\partial u_i'}{\partial x_m}}$を乗じ,$i$,$j$について和をとり,両辺$\nu$を掛けてアンサンブル平均する.
\end{itemize}
ことで$\varepsilon$の支配方程式を得る.\\

\cite{KidaYanase}の方法で計算する.
\begin{align*}
  \Big( \frac{\partial}{\partial t} +\bar{u}_k \frac{\partial}{\partial x_k} \Big) \frac{\partial u_i'}{\partial x_m} = &-\frac{\partial^2 \bar{u}_i}{\partial x_k \partial x_m} u_{k\textcircled{\scriptsize 1}}' -\frac{\partial \bar{u}_k}{\partial x_m} \frac{\partial u_i'}{\partial x_k}_{\textcircled{\scriptsize 1}} -\frac{\partial \bar{u}_i}{\partial x_k} \frac{\partial u_k'}{\partial x_m}_{\textcircled{\scriptsize 1}} \\
  &-\frac{\partial}{\partial x_k \partial x_m} (u_i' u_k' -\overline{u_i'u_k'})_{\textcircled{\scriptsize 2}} -\frac{1}{\rho} \frac{\partial^2 p'}{\partial x_i \partial x_m}_{\textcircled{\scriptsize 3}} +\nu \frac{\partial^3 u_i'}{\partial x_k \partial x_k \partial x_m}_{\textcircled{\scriptsize 4}}
\end{align*}
\begin{align*}
  \textcircled{\scriptsize 1} &: -\frac{\partial^2 \bar{u}_i}{\partial x_k \partial x_m} u_k' \frac{\partial u_i'}{\partial x_m} -\frac{\partial \bar{u}_k}{\partial x_m} \frac{\partial u_i'}{\partial x_k} \frac{\partial u_i'}{\partial x_m} -\frac{\partial \bar{u}_i}{\partial x_k} \frac{\partial u_k'}{\partial x_m} \frac{\partial u_i'}{\partial x_m} \\
    &~~~~ \to -2 u_k' \frac{\partial^2 \bar{u}_i}{\partial x_k \partial x_m} \frac{\partial u_i'}{\partial x_m} -2 \frac{\partial \bar{u}_i}{\partial x_k} \Big( \frac{\partial u_i'}{\partial x_m} \frac{\partial u_k'}{\partial x_m} +\frac{\partial u_m'}{\partial x_k} \frac{\partial u_m'}{\partial x_i} \Big) \\
  \textcircled{\scriptsize 2} &: -\frac{\partial}{\partial x_k \partial x_m} (u_i' u_k' -\overline{u_i'u_k'})  \frac{\partial u_i'}{\partial x_m} \\
    &~~~~ \to -2 \frac{\partial u_i'}{\partial x_m}  \frac{\partial u_k'}{\partial x_m}  \frac{\partial u_i'}{\partial x_k} - \frac{\partial}{\partial x_k} \Big\{ u_k' \Big(  \frac{\partial u_i'}{\partial x_m} \Big)^2 \Big\} +2 \frac{\partial u_i'}{\partial x_m}  \frac{\partial^2 \overline{u_i'u_k'}}{\partial x_m \partial x_k} \\
  \textcircled{\scriptsize 3} &: -\frac{1}{\rho} \frac{\partial^2 p'}{\partial x_i \partial x_m} \frac{\partial u_i'}{\partial x_m} \\
    &~~~~ \to -\frac{1}{\rho} \frac{\partial}{\partial x_i} \Big( \frac{\partial p'}{\partial x_m} \frac{\partial u_i'}{\partial x_m} \Big) \\
  \textcircled{\scriptsize 4} &: \nu \frac{\partial^3 u_i'}{\partial x_k \partial x_k \partial x_m} \frac{\partial u_i'}{\partial x_m} \\
    &~~~~ \to \nu \frac{\partial^2}{\partial x_k} \Big( \frac{\partial u_i'}{\partial x_m} \Big)^2 -2\nu \Big( \frac{\partial^2 u_i'}{\partial x_m \partial x_k} \Big)^2
\end{align*}

これに$\nu$を乗じアンサンブル平均を取ると,$\varepsilon$の支配方程式が得られる.(未確認)
\begin{equation}
  \frac{\overline{\mathrm{D}} \varepsilon}{\overline{\mathrm{D}} t} = P_\varepsilon -\Phi_\varepsilon +\frac{\partial J_{(\varepsilon)ij}}{\partial x_j}
  \label{eq:epsilon_equation} \tag{6.39}
\end{equation}
\begin{align}
  P_\varepsilon = &-2\nu \Big( \overline{\frac{\partial u_i'}{\partial x_k}\frac{\partial u_j'}{\partial x_k}} +\overline{\frac{\partial u_k'}{\partial x_i}\frac{\partial u_k'}{\partial x_j}} \Big) \frac{\partial \bar{u}_j}{\partial x_i} -2\nu \overline{u_i' \frac{\partial u_j'}{\partial x_k}} \frac{\partial^2 \bar{u}_j}{\partial x_i \partial x_k} \nonumber \\
    &-2\nu \overline{ \frac{\partial u_j'}{\partial x_i} \frac{\partial u_j'}{\partial x_k} \frac{\partial u_i'}{\partial x_k} } \tag{6.40} \\
  \Phi_\varepsilon = &2\nu^2 \overline{ \frac{\partial^2 u_i'}{\partial x_j \partial x_k} \frac{\partial^2 u_i'}{\partial x_j \partial x_k} } \tag{6.41} \\
  J_{(\varepsilon)j} = &-\nu \overline{ \frac{\partial u_i'}{\partial x_k}\frac{\partial u_i'}{\partial x_k}u_j' } -\frac{\nu}{\rho} \overline{ \frac{\partial u_j'}{\partial x_i}\frac{\partial p'}{\partial x_i} } +\nu \frac{\partial \varepsilon}{\partial x_j} \tag{6.42~-~6.44}
\end{align}

以上より2パラメータ$k$,$\varepsilon$の支配方程式が得られた.
これらの次元は$[k]=[L^2/T^2]$,$[\varepsilon]=[L^2/T^3]$である.

\ref{subsec:SmagorinskyConstant}節より,渦動粘性係数の次元は$[\nu_T] =[\mathrm{L}^2/\mathrm{T}] =[\mathrm{L^4}/\mathrm{T^4}][\mathrm{T}^3\mathrm{L}^2]$と書くことができる.
そこで,無地限定数$C_\mu$を導入し,$k$と$\varepsilon$によって次のように書ける.
\begin{equation}
  \nu_T = C_\mu \frac{k^3}{\varepsilon}
  \label{eq:k-epsilon_nuT} \tag{6.38}
\end{equation}

これによって$\nu_T$を二つのパラメータで評価することができて,Eq. \ref{eq:EddyViscosityNSEquation}が閉じた.\\

$k$,$\varepsilon$の支配方程式を並べる.
\begin{align}
  \frac{\overline{\mathrm{D}} k}{\overline{\mathrm{D}} t} &= P_k -\varepsilon +\frac{\partial J_{(k)ij}}{\partial x_j}
  \tag{6.30} \\
  \frac{\overline{\mathrm{D}} \varepsilon}{\overline{\mathrm{D}} t} &= P_\varepsilon -\Phi_\varepsilon +\frac{\partial J_{(\varepsilon)ij}}{\partial x_j}
  \tag{6.39}
\end{align}
ここで,\underline{これらの方程式も3次以上の項を含み,Reynolds応力の方程式として閉じていない}ことに注意する.
これを解消するため,
拡散項$J_{(\cdot)ij}$に勾配拡散近似を導入する.
\begin{equation}
  J_{(\cdot)ij} = \Big( \frac{\nu_T}{\sigma_\cdot} +\nu \Big) \frac{\partial ~\cdot}{\partial x_j}
  \label{eq:GradientDiffusionApproximation} \tag{6.37, 6.46}
\end{equation}
また,$\varepsilon$方程式の生産項と消滅項を一括してモデル化する.
\begin{equation}
  P_\varepsilon -\Phi_\varepsilon = (C_{\varepsilon1}P_k -C_{\varepsilon2}\varepsilon) \frac{\varepsilon}{k}
  \label{eq:Modelinepsilon} \tag{6.45}
\end{equation}

以上をまとめて,$k$,$\varepsilon$の支配方程式は次のようになる.
\begin{align}
  \frac{\overline{\mathrm{D}} k}{\overline{\mathrm{D}} t} &= P_k -\varepsilon +\frac{\partial}{\partial x_j} \Big[ \Big( \frac{\nu_T}{\sigma_k} +\nu \Big) \frac{\partial k}{\partial x_j} \Big]
  \tag{6.50} \\
  \frac{\overline{\mathrm{D}} \varepsilon}{\overline{\mathrm{D}} t} &= (C_{\varepsilon1}P_k -C_{\varepsilon2}\varepsilon) \frac{\varepsilon}{k} +\frac{\partial}{\partial x_j} \Big[ \Big( \frac{\nu_T}{\sigma_\varepsilon} +\nu \Big) \frac{\partial \varepsilon}{\partial x_j} \Big]
  \tag{6.51}
\end{align}

\printbibliography[title=参考文献]

\end{document}
