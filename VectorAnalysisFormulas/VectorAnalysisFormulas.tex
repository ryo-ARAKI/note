\documentclass[12pt,dvipdfmx,svgnames,a4paper,uplatex]{ujreport}
% +++++++++++++++++++++++++++++++++++++++++++
% パッケージの導入
% +++++++++++++++++++++++++++++++++++++++++++
%
% ===========================================
% 原稿設定
% ===========================================
% \usepackage{hyoshi}  % 表紙用スタイルファイル
% 原稿のサイズ
\usepackage[top=25truemm,bottom=25truemm,left=25truemm,right=25truemm]{geometry}
\usepackage{relsize}
%
% ===========================================
% 図・表関係
% ===========================================
\usepackage{graphicx}
\graphicspath{{./pics}}  % \includegraphicsで参照するディレクトリ
%
% ===========================================
% 参考文献
% ===========================================
\usepackage[url=false,isbn=false,doi=false]{biblatex}
\addbibresource{./bib_textbooks.bib}  % 教科書など
% \addbibresource{./bib_articles.bib}  % 論文など
%
% ===========================================
% 独自スタイルの導入
% ===========================================
\usepackage{/home/ryo/.config/LaTeX/mystyle}
\renewcommand{\prechaptername}{第}  % mystyle.styに書くとbeamerでも読み込まれるため,エラーとなる
\renewcommand{\postchaptername}{章}
%
% ===========================================
% 表紙の記述
% ===========================================
\title{\underline{ベクトル解析の公式(の導出)まとめ}}
\author{荒木~亮}
\date{\today}

% +++++++++++++++++++++++++++++++++++++++++++
% 本文
% +++++++++++++++++++++++++++++++++++++++++++
\begin{document}
\maketitle
% \tableofcontents
% \clearpage

三次元Euclid空間\(\mathbb{R}\)上で定義されたベクトル\(\vb*{A}, \vb*{B}, \dots\)を考える.

\begin{shadebox}
  ベクトル\(\vb*{A}\)と\(\vb*{B}\)の外積は,
  \begin{equation}
    (\vb*{A} \times \vb*{B})_i = \epsilon_{ijk} A_j B_k
  \end{equation}
  のように書ける.
  ここで,\(\epsilon_{ijk}\)はEddington's epsilonやLevi-Civita symbolと呼ばれるテンソルで,
  \begin{empheq}[left={\epsilon_{ijk} = \empheqlbrace}]{align}
    1 &\qq{if \((i,j,k)\) is an even permutation of \((1,2,3)\)} \\
    -1 &\qq{if \((i,j,k)\) is a odd permutation of \((1,2,3)\)} \\
    0 &\qq{else}
  \end{empheq}
  と定義される.
\end{shadebox}

\begin{itembox}[l]{\(\
  \epsilon_{ikl} \epsilon_{lmn} \
  = \delta_{im} \delta_{kn} \
  - \delta_{in} \delta_{km} \
\)}
  \begin{equation}
    \text{L.H.S.} \
      = \epsilon_{ik1} \epsilon_{1mn} \
      + \epsilon_{ik2} \epsilon_{2mn} \
      + \epsilon_{ik3} \epsilon_{3mn}
  \end{equation}
  ここで\((i,k) = (1,2)\)を代入してみると,
  \begin{empheq}{align}
    \text{L.H.S.} \
      &= \epsilon_{121} \epsilon_{1mn} \
      + \epsilon_{122} \epsilon_{2mn} \
      + \epsilon_{123} \epsilon_{3mn} \\
      &= \epsilon_{3mn} \\
    \text{R.H.S} &= \delta_{1m} \delta_{2n} - \delta_{1n} \delta_{2m}
  \end{empheq}
  これはたしかに等値できる.
  同様のことが\((i,k)\)の他の組み合わせについても成立するので,与式は成立する.\(\square\)
\end{itembox}

\begin{itembox}[l]{\(\epsilon_{ikl} \epsilon_{klm} = 2\delta_{im}\)}
  \begin{empheq}{align}
    \text{L.H.S.} \
      &= -\epsilon_{ikl} \epsilon_{lkm} \\
      &= -(\delta_{ik} \delta_{km} - \delta_{im} \delta_{kk}) \\
      &= -(\delta_{im} - 3\delta_{im}) \\
      &= 2\delta_{im}
  \end{empheq}
  より示せた.\(\square\)
\end{itembox}

\begin{itembox}[l]{\(\
  \div(\vb*{A} \times \vb*{B}) \
  = \vb*{B} \vdot (\curl{\vb*{A}}) \
  - \vb*{A} \vdot (\curl{\vb*{B}}) \
\)}
  成分表記で調べて,
  \begin{empheq}{align}
    \partial_i (\epsilon_{ijk} A_j B_k) \
      &= B_k \epsilon_{ijk} \partial_i A_j + A_j \epsilon_{ijk} \partial_i B_k \\
      &= B_k \epsilon_{kij} \partial_i A_j - A_j \epsilon_{jik} \partial_i B_k \\
      &= \text{R.H.S.}
  \end{empheq}
  で示せた.\(\square\)
\end{itembox}

\begin{itembox}[l]{\(\
  \grad(\vb{A} \vdot \vb*{B}) \
  = \vb*{A} \times (\curl{\vb*{B}}) \
  + \vb*{B} \times (\curl{\vb*{A}}) \
  + (\vb*{B} \vdot \grad) \vb*{A} \
  + (\vb*{A} \vdot \grad) \vb*{B} \
\)}
  右辺の\(i\)成分を調べて,
  \begin{empheq}{align}
    [\text{R.H.S.}]_i \
      &= \epsilon_{ijk} A_j (\epsilon_{klm} \partial_l B_m) \
      + \epsilon_{ijk} B_j (\epsilon_{klm} \partial_l A_m) \
      + B_j \partial_j A_i \
      + A_j \partial_j B_i \\
      &= \epsilon_{ijk} \epsilon_{klm} (A_j \partial_l B_m + B_j \partial_l A_m) + \dots \\
      &= (\delta_{il} \delta_{jm} - \delta_{im} \delta_{jl}) (A_j \partial_l B_m + B_j \partial_l A_m) \
      + \dots \\
      &=(A_j \partial_i B_j + B_j \partial_i A_j) \
      - (A_j \partial_j B_i + B_j \partial_j A_i) \
      + \dots \\
      &= A_j \partial_i B_j + B_j \partial_i A_j \\
      &= \partial_i (A_j B_j) \\
      &= [\text{L.H.S.}]_i
  \end{empheq}
  より示せた.\(\square\)
\end{itembox}

\begin{itembox}[l]{\(\
  \curl(\vb*{A} \times \vb*{B}) \
  = \vb*{A} (\div{\vb*{B}}) \
  - \vb*{B} (\div{\vb*{A}}) \
  + (\vb*{B} \vdot \grad) \vb*{A} \
  - (\vb*{A} \vdot \grad) \vb*{B} \
\)}
  \begin{empheq}{align}
    [\text{L.H.S.}]_i \
      &= \epsilon_{ijk} \partial_j (\epsilon_{klm} A_l B_m) \\
      &= \epsilon_{ijk} \epsilon_{klm} \partial_j (A_l B_m) \\
      &= (\delta_{il} \delta_{jm} - \delta_{im} \delta_{jl}) \partial_j (A_l B_m) \\
      &= \partial_j (A_i B_j) \
      - \partial_j (A_j B_i) \\
      &= A_i \partial_j B_j \
      + B_i \partial_j A_j \
      - B_i \partial_j A_j \
      - A_j \partial_j B_i \\
      &= [\text{R.H.S.}]_i
  \end{empheq}
  より示せた.\(\square\)
\end{itembox}

\begin{itembox}[l]{\(\
  \curl(\curl{\vb*{A}}) = \grad(\div{\vb*{A}}) - \laplacian \vb*{A}
\)}
  \begin{empheq}{align}
    [\text{L.H.S.}]_i \
      &= \epsilon_{ijk} \partial_j (\epsilon_{klm} \partial_l A_m) \\
      &= \epsilon_{ijk} \epsilon_{klm} \partial_j (\partial_l A_m) \\
      &= (\delta_{il} \delta_{jm} - \delta_{im} \delta_{jl}) \partial_j (\partial_l A_m) \\
      &= \partial_j (\partial_i A_j) - \partial_j (\partial_j A_i) \\
      &= \partial_i (\partial_j A_j) - \partial_j \partial_j A_i \\
      &= [\text{R.H.S.}]_i
  \end{empheq}
\end{itembox}

\end{document}
